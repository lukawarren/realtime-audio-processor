\subsubsection{Student 1}
{
\centering
\fbox{\begin{minipage}{15cm}
		\begin{center}
			{\huge Luka's Questionnaire Form}
		\end{center}
		
		For my A-level Computer Science coursework  I am writing a program that allows users to easily apply various audio effects to live songs, in order to make experimenting with music and creating remixes easier. In other words, I am writing a program where you can "remix" music as you listen to it, so that experimentation can be done quickly and without hassle. In order to create the best possible software, I would like your opinion on what makes a remix good, and what features you would expect my software to have. Please answer the questions below by editing this document in a concise and comprehensible manner, then email me your responses.
		
		\paragraph{Questions}
		\begin{enumerate}
			\item Why do you sometimes prefer a song's remix?
			\item How does a remix typically differ from the original song?
			\item My program is meant to lower the barrier of entry for editing audio as much as possible. How can I further enhance ease-of-use?
			\item Are there any other features  you would like in a real-time audio editing program to assist in "remixing" music?
			\item How do you currently listen to music?
		\end{enumerate}
		
		\paragraph{Responses:}
		\begin{enumerate}
			\item Videos on TikTok are only about 30 seconds long. It's good to speed up a song because otherwise you couldn't enjoy the full chorus. I also think music played at different speeds makes it sound new and more interesting.
			\item It's usually faster or slower, and people like to change the bass as well so it sounds better.
			\item I've tried editing audio before but I always get confused with all the options. I like the idea of it being live because then I can instantly hear what difference my choices are making. But if I had to pick one new feature, I'd want one of those music visualisations as well. It's always very easy to understand if a song has lots of bass, for example, because you can see the bass being visualised. If I was doing things like changing the bass, being able to see this change (as well as hear it) would be really useful.
			\item See above.
			\item I use Spotify. You can't do any audio editing in that though. It's a shame I can't change the speed.
		\end{enumerate}
		\bigskip \bigskip \bigskip
\end{minipage}}
}

\pagebreak
\subsubsection{Student 2}
{
	\centering
	\fbox{\begin{minipage}{15cm}
			\begin{center}
				{\huge Luka's Questionnaire Form}
			\end{center}
			
			For my A-level Computer Science coursework  I am writing a program that allows users to easily apply various audio effects to live songs, in order to make experimenting with music and creating remixes easier. In other words, I am writing a program where you can "remix" music as you listen to it, so that experimentation can be done quickly and without hassle. In order to create the best possible software, I would like your opinion on what makes a remix good, and what features you would expect my software to have. Please answer the questions below by editing this document in a concise and comprehensible manner, then email me your responses.
			
			\paragraph{Questions}
			\begin{enumerate}
				\item Why do you sometimes prefer a song's remix?
				\item How does a remix typically differ from the original song?
				\item My program is meant to lower the barrier of entry for editing audio as much as possible. How can I further enhance ease-of-use?
				\item Are there any other features  you would like in a real-time audio editing program to assist in "remixing" music?
				\item How do you currently listen to music?
			\end{enumerate}
			
			\paragraph{Responses:}
			\begin{enumerate}
				\item I get bored listening to the same songs over and over, but hearing a remix makes it sound new and exciting.
				\item Usually there's a small echo that's added (I think it's called reverb). Also there's these crackles that get added to Lo-Fi music called noise that I think makes music quite calming. I'd really like to be able to add noise to my playlist. People like to speed up songs slightly as well, especially on TikTok or Instagram Reels. On YouTube there's lots of versions of songs where they've bass-boosted it as well. 
				\item I want to be able to change what effects are applied, and also change the speed of the music. I don't want to have to pause the music, apply an effect, wait for it to be processed, then unpause. So I think you making it live is a good idea.
				\item I hope the program supports letting me play my entire playlist - that way I can just add a few effects and sit back as it changes my entire music collection!
				\item I use Apple Music but I also have my songs on my laptop as audio files in case my Wi-Fi runs out. 
			\end{enumerate}
			\bigskip \bigskip \bigskip
	\end{minipage}}
}

\pagebreak
\subsubsection{Student 3}
{
	\centering
	\fbox{\begin{minipage}{15cm}
			\begin{center}
				{\huge Luka's Questionnaire Form}
			\end{center}
			
			For my A-level Computer Science coursework  I am writing a program that allows users to easily apply various audio effects to live songs, in order to make experimenting with music and creating remixes easier. In other words, I am writing a program where you can "remix" music as you listen to it, so that experimentation can be done quickly and without hassle. In order to create the best possible software, I would like your opinion on what makes a remix good, and what features you would expect my software to have. Please answer the questions below by editing this document in a concise and comprehensible manner, then email me your responses.
			
			\paragraph{Questions}
			\begin{enumerate}
				\item Why do you sometimes prefer a song's remix?
				\item How does a remix typically differ from the original song?
				\item My program is meant to lower the barrier of entry for editing audio as much as possible. How can I further enhance ease-of-use?
				\item Are there any other features  you would like in a real-time audio editing program to assist in "remixing" music?
				\item How do you currently listen to music?
			\end{enumerate}
			
			\paragraph{Responses:}
			\begin{enumerate}
				\item Honestly some songs are just too slow! I just want to get to the chorus but no - I have to wait! I know it's silly but when a song's faster I feel you get more out of it faster. Your project would really be useful to me in that regard.
				\item I think it differs in tempo and pitch.
				\item Well as long as it's real-time I think it should already be very easy to use and understand. But make sure it's not laggy or anything because my laptop isn't very fast. Also adding echoes to make music sound like it's distant is something I really like, and I don't know any way to do that without complicated editing software.
				\item I really like the YouTube music visualisations where you can see the music react. I don't know how to describe it but like you can see the bass part, the vocals part, etc. and I think seeing how that changes when you alter the music would be really useful.
				\item Spotify on my phone and Apple Music on my laptop.
			\end{enumerate}
			\bigskip \bigskip \bigskip
	\end{minipage}}
}