\section{Evaluation}
\paragraph{}
Now that I have finished developing my software project, it is time to evaluate if I have achieved my intended overall goal. Naturally the goal of the project was realised by distilling the various requirements into discrete objectives and attempting to accomplish each respectively, but two key questions must be stressed:
\begin{enumerate}
	\item How well were these objectives met?
	\item Did the completion of these objectives result in properly addressing the initial problem?
\end{enumerate}

\paragraph{}
Furthermore, whilst the initial aim of the project may have been met, to what extent could the software be improved? Consider, for example, that whilst a user may be able to create many remixes in realtime using this software, highlighting that the goal was broadly met, there is a possibility that more nuanced or bizarre effects, present sometimes in other remixes, are not supported currently within the software.

\paragraph{}
Given the somewhat subjective nature of all these questions, I have decided the best way to reach a thorough conclusion is to interview the same target audience that I first interviewed in the analysis section. As it was their opinions and decisions that ultimately guided the project, they are well positioned to critique the final product by contrasting it with the vision they had in their own minds.

\pagebreak
\subsection{Evaluating the Completion of Objectives}
\paragraph{}
Before I can evaluate how well the completion of the objectives resulted in the initial requirement being fulfilled,  I must naturally first evaluate if the objectives themselves were completed in the first place.

\paragraph{}
Recall that in the design section (section 2), I deliberately grouped tests by objective, in such a manner that the required functionality of each objective was intended to be tested by the tests contained next to it. For example, tests 1.1 - 1.7, upon competition, were designed to assure that objective 1 had passed. 

\paragraph{}
\textbf{After careful debugging and iteration, the program has by now reached a state where all tests pass.} Hence all that remains is to evaluate, for each objective, if the suite of tests associated with it can be reasonably said to assure the objective is met.

\subsubsection{Evaluating Objective 1}
\paragraph{Description} "The user must be able to load a collection of audio files known as a “playlist” and then play the audio
files contained within, organised alphabetically, as is the custom in audio-listening applications."

{
	\renewcommand{\arraystretch}{1.7}
	\begin{table}[h!]
		\begin{center}
			\begin{tabularx}{1.0 \textwidth} {
					| >{\raggedright\arraybackslash}X
					| >{\raggedright\arraybackslash}X
					| >{\raggedright\arraybackslash}X
					|
				}
				\hline
				Aspect of Objective & Tests  & All Tests Passed? \\
				
				\hline
				"be able to load... playlist[s]" & 1.1, 1.2 & Yes\\
				
				\hline
				"play the audio files within" & 1.4, 1.6 & Yes\\
				
				\hline
				"organised alphabetically" & 1.3 & Yes\\
				
				\hline
			\end{tabularx}
		\end{center}
	\end{table}
}

Recall also that tests 1.1, 1.2 and 1.4 were tested not just with correct, "non-corrupt" audio files but also a variety of invalid data, including non-audio files, non-existent files and corrupt audio files. Hence I can also be sure that the objective is met in the most robust sense possible, in that playlists can still be continued to be loaded, for example, even after the user has tried to load an invalid one.

\paragraph{}
Hence as each part of the objective has been thoroughly tested, and each of these tests has passed, I can say with confidence that objective 1 has been met.

\pagebreak
\subsubsection{Evaluating Objective 2}
\paragraph{Description} "The user must be able to visualise the current audio being played in the frequency domain (i.e. visualise
the frequencies)"

\paragraph{}
Through tests 2.1 - 2.7, which all passed, it was ascertained that various frequencies could be visualised correctly, both in isolation and in combination with each other. For example, not only can a sine wave of 1000 Hz or 10,000 Hz be visualised, but an audio file consisting of both can also be visualised. This was tested with increasing numbers of sine waves, such that the conclusion reached was that the visualisation of audio with the frequency domain was correct regardless of the complexity of the audio. Furthermore, other tests, such as tests 3.1 and 3.2 (which altered audio in the frequency domain) resulted in the correct change to the visualisation, further reinforcing that the frequency domain was being displayed correctly.

\paragraph{}
As it has therefore been proven that a variety and combination of frequencies can be visualised, and that the frequency domain of a song can be seen to visually change in the correct way when modified, I have therefore arrived at the conclusion that objective 2 has been met. In no way was it ever apparent that the frequency domain visualisation produced incorrect or unexpected results, and indeed even when playing real-world music the visualisation looked entirely plausible\footnote{
	The visualisation of music itself was not directly tested due to the challenge of identifying what a "correct" visualisation would look like - humans cannot, unfortunately, visualise audio in the frequency domain in their heads. This is why simple sine waves were used instead, which thankfully can be envisioned. However, in other unrelated tests where proper music was played, the visualisations displayed looked very plausible, with peaks seen corresponding to stand-out frequencies, and vocals and bass frequencies clearly shown. This is not a test unto itself but rather a "sanity check" that the results of "sine wave tests" can indeed be extrapolated to more general audio.
}.

\pagebreak
\subsubsection{Evaluating Objective 3}
\paragraph{Description} "The user must be able to modify the audio’s frequency domain (i.e. selectively modify frequencies such
as by reducing the bass)"

\paragraph{}
Three tests were conducted to test objective 3. Tests 3.1 and 3.2, which both passed, tested if specific frequencies (the bass and treble respectively) could have their amplitudes reduced\footnote{
	These frequency ranges were specifically picked as they are easy to verify be ear, and also represent extremely common modifications made in song remixes.
}. Test 5.3 tested a range of frequencies more generally, and its attached evidence showed that any given frequency range could have its amplitude either increased or decreased, each time resulting in the correct auditory change. In addition, this was verified by means of the audio visualisation, as in each instance the frequencies visualised adapted to the way the audio had been modified in the expected way.

\paragraph{}
Thus, as all tests passed, it can be understood with confidence that the audio's frequency domain can be modified correctly. One has only listen, for example, to the evidence of tests 3.1 or 3.2 to understand this. Hence I am of the belief that objective 3 was firmly met.

\pagebreak
\subsubsection{Evaluating Objective 4}
\paragraph{Description} "The user must be able to apply additional “audio effects” to further enhance the music: echo, volume
adjustment and noise"

{
	\renewcommand{\arraystretch}{1.7}
	\begin{table}[h!]
		\begin{center}
			\begin{tabularx}{1.0 \textwidth} {
					| >{\raggedright\arraybackslash}X
					| >{\raggedright\arraybackslash}X
					| >{\raggedright\arraybackslash}X
					| >{\raggedright\arraybackslash}X
					| >{\raggedright\arraybackslash}X
					|
				}
				\hline
				Audio Effect & Test & Target Audience deemed effect correct? & Subject specialist deemed effect correct? & Test passed? \\
				
				\hline
				Echo & 4.1 & Yes & Yes & Yes \\
				
				\hline
				Volume Adjustment & 4.2 & Yes & Yes & Yes \\
				
				\hline
				Noise & 4.3 & Yes & Yes & Yes \\
				
				\hline
			\end{tabularx}
		\end{center}
	\end{table}
}

All audio effects mentioned within the objective were verified by a select group of listeners, including a subject specialist. In each case, it was deemed that the audio effect produced the intended result. Hence, as all tests passed, it can deemed that the objective passed.

\pagebreak
\subsubsection{Evaluating Objective 5}
\paragraph{Description} "The user must be able to configure these “audio effects” individually, yet also apply pre-made “presets”
to quickly reach a desired effect"

{
	\renewcommand{\arraystretch}{1.7}
	\begin{table}[h!]
		\begin{center}
			\begin{tabularx}{1.0 \textwidth} {
					| >{\raggedright\arraybackslash}X
					| >{\raggedright\arraybackslash}X
					| >{\raggedright\arraybackslash}X
					|
				}
				\hline
				Aspect of Objective & Tests  & All Tests Passed? \\
				
				\hline
				Configure audio effects individually & 5.1, 5.2, 5.3 & Yes\\
				
				\hline
				Apply pre-made "presets" to quickly reach a desired effect & 5.4 & Yes\\
				
				\hline
			\end{tabularx}
		\end{center}
	\end{table}
}

Through the passing of tests it was understood that each audio effect could be configured in the appropriate manner, such that the correct auditory change was heard. It was also shown under test 5.4 that a variety of presents were able to be applied. Many different "desired effects" could be reached, ranging from "far away room" to "remove bass". All these "desired effects" were reported to be achieved by those interviewed, including the subject specialist.

\paragraph{}
Thus by all accounts the objective was met.

\pagebreak
\subsubsection{Evaluating Objective 6}
\paragraph{Description} "The system must run in real-time on an average school computer"

\paragraph{}
Great effort was taken to ensure the performance characteristics of the code aligned with the definition of "real-time" chosen in the design section (section 2). After measuring performance in tests 6.1 and 6.2, it finally transpired that performance was actually better than expected, in that  even on hardware much weaker than an average school computer, all computation could be performed in less time than was allotted before the program could no longer be considered real-time. In other words, testing showed that performance was better than required.  Tests 7.3 and 7.4, whilst more related to the altering of playback speed, nevertheless also showed that speeding up or slowing down music had a negligible performance cost.

\paragraph{}
Recall that the definition of real-time employed in this project (and in computer graphics, audio processing, etc.) refers to there being an imperceivable delay. Research showed that for audio this delay had to be under 3ms, and for visuals the delay had to be ideally under 16.6ms\footnote{See "Testing Objective 6" under section 2}. As all relevant tests passed with times under these, both these performance targets were met.

\paragraph{}
Hence the overall objective was met.

\pagebreak
\subsubsection{Evaluating Objective 7}
\paragraph{Description} "The user must be able to alter the speed at which audio is played"

\paragraph{}
All tests for objective 7 passed. Hence it was shown that audio could both be sped up and slowed down at will, such that objective 7 was met.

\pagebreak
\subsection{Interview with Original Target Audience}
\paragraph{}
In order to know if the target audience felt that the aim of the project (in their minds, and for their needs) had been met, another interview was conducted, using the same group as before in section 1.  As it was their opinions and decisions that ultimately guided the project, they are well positioned to critique the final product by contrasting it with the vision they had in their own minds. The following questions were asked:

\begin{enumerate}
	\item After using the software for yourself, do you think you would be able to use it to create the majority of "song remixes" you hear online?
	\item Is the software intuitive and easy to use?
	\item Are you prevented, in any way,  from replicating any song remixes you have heard? If so, what features does this software lack that would help resolve this issue?
	\item Was the performance of the software acceptable, to the extent that you were able to edit audio in real-time (i.e live)?
\end{enumerate}

For the benefit of the reader, the responses have been copied into the document below to aid readability. However, \textbf{screenshots of the raw responses themselves can be viewed in the appendix} (under "Evaluation Interview Responses").

\subsection{Student 1 Response}
\begin{enumerate}
	\item Most of the remixes I hear are sped up and bass boosted. If I adjust the playback speed and use your "remove treble" preset, I can get music to sound like that very easily. I've heard other song remixes where they add reverb and at first I wasn't sure how to do that, but your "far away room" [preset] let me do that without configuring the audio effects combinations myself. So yes overall I think I would be able to create 99 percent of remixes I've heard.
	\item I think the software was very intuitive - I immediately understood what the bars meant (like the visualiser in the middle). It was very easy to pause music, play, etc. and change the speed too - the controls were very familiar. I don't think I encountered any undue difficulties.
	\item No, I think all the remix effects I'd want to create can be made if you combine all the audio effects you have carefully together.
	\item Yes, the audio was real-time, because it didn't lag or anything. The program ran at 60 FPS so it wasn't laggy.
\end{enumerate}

\subsection{Student 2 Response}
\begin{enumerate}
	\item Well I mostly listen to Lo-Fi and your software literally had two presets just for that so I think I'd be able to remake whatever remixes I wanted based on what I usually hear.
	\item I was a bit annoyed at one point because I wanted to remix a single song (like just open the WAV file) but I had to create an entire playlist first. Everything else was fine (was never confused about anything) but I was annoyed by this issue because sometimes you don't need an entire playlist. 
	\item I don't have much experience with making remixes myself because before your project came along it wasn't something I thought about but off the top of my head, I can't think of any other ways in which people change songs on social media. So I think you've covered it.
	\item It ran fine for me on my laptop so yes it was real-time. The audio didn't slow down, etc.
\end{enumerate}

\subsection{Student 3 Response}
\begin{enumerate}
	\item Honestly all the ones online all sound very similar to me, but thinking about how they all sound then yes, I think using your program I could make all of them if I wanted to.
	\item The program was very easy to use - I instantly knew what to do.
	\item Well I can make the remixes themselves, but how do I share them? Like if I wanted to put it on TikTok, I'd need to export it to an audio file somehow. I think that's something you forgot about - I need to be able to share remixes as well as make them!
	\item Performance was fine.
\end{enumerate}

\pagebreak
\subsection{Final Evaluation using Tests and Interview Responses}
\subsubsection*{Contrasting Objectives and Interview Responses}
After evaluating the objectives above, it was made clear that they were all successfully met by virtue of passing the tests prescribed (see above). Thanks to this, the user can now load playlists, visualise audio in the frequency domain, apply various audio effects and more, all for the purpose of creating song remixes. However, the question remains: does this completion of these objectives mean that the overall aim of the project has been met?

\paragraph{}
Based off the responses from the evaluation interview conducted, users were generally positive, indicating that all the remixes they had personally wished to create could indeed be replicated within the software. Indeed, not a single interviewee highlighted any missing features that prevented them from creating song remixes. As such, there is an extremely strong case that the project's goal, of allowing the creation of song remixes in real-time, has been met. Whilst there are most likely various other audio effects that remixes sometimes employ, they do not appear to be sufficiently common to pop-up in any interviews conducted, and as such the goal can said to be met for the vast majority  of users. With regards to the "real-time" stipulation of the project, all users agreed that, based of the performance they experienced, they did indeed consider the project to be "real-time". Thus, based off interview responses, I would argue that the majority of people would be able to create any remix of their choosing using the software. As an added positive, all users felt the software was intuitive and unconfusing, which highlights another barrier to creating song remixes that my software has removed for many.

\paragraph{}
Thus it can be seen that by implementing the objectives arrived upon in section 1, the overall aim of the project has been met - users can indeed create remixes of songs in real-time, with every common "remix effect" being covered for the target audience interviewed.

\subsubsection*{Room for Growth}
Whilst the overall project aim has been met, two users identified key issues with the program that, whilst not interfering with the core aim of the project itself, nevertheless would warrant substantial reconsideration if it was to be revisited. 

\paragraph{}
Student 2 felt the work-flow of the program was inconvenient when one only wanted to play a single file, as being forced to "create an entire playlist" was cumbersome.  Instead, he thought that there should have been an option to play a single audio file in isolation. In retrospect, this appears to be an obvious omission, as there will be many times when a user simply has no need for multiple songs to be remixed, and as such forcing them to create a "playlist" of just one file is likely to confuse the user and potentially waste their time. If I were to revisit the project, I would certainly attempt to solve this by altering the "start window" of the program to add an option to select a single audio file. Upon selection of this audio file, a "single-file playlist" could then be created \textit{in memory only} (i.e. not saved to disk), allowing the rest of the code (that depends on playlists) to remain unmodified. I don't think this would be especially difficult and would greatly improve usability.

\paragraph{}
Student 3 felt strongly that the project lacked the ability to export created remixes for sharing with others. There is indeed merit to this: remixes gain great popularity online on social media websites such as TikTok, but how can they be shared if the program lacks the ability to export said remixes? Solving this would likely not be a great challenge - if a "record button" was pressed, one could simply begin to record all processed audio played by the program into a buffer. Then when the recording ended, this audio buffer could be saved to disk as a WAV file for consumption by others. I would certainly implement this feature, as I can imagine many users, after creating remixes, would wish strongly to share their artistic work with others and feel frustrated at the software's current inability to do this.

\paragraph{}
Thus both these features, highlighted by users through dialogue, showcase the potential for this project to grow in the future, should it be revisited. However, it should be stressed that neither issue prevents the project's overall goal from being met in any way: student 2's issue is only concerned with convenience and does not detract from one's ability to create song remixes, and student 3's issue, whilst pressing, is more concerned with the \textit{sharing} of remixes than their \textit{creation}. Thus neither issue impacts one's ability to create song remixes. However, it is true that there is more to remixes than creating them; many people evidently enjoy sharing them too. Hence revisiting this project would certainly have to involve reconsidering the broader context in which the software is to be used (i.e. not just being able to make a remix, but what to do afterwards, including sharing it). Allowing the playing of single audio files (as opposed to playlists) is a simple fix that does not require the same level of reflection, but should also nevertheless be considered.

\subsubsection*{Conclusion}
To conclude, careful dialogue with the project's target audience, in addition to a review of the objectives and how they reflect upon the completion of the project's goal as a whole has led me to thee conclusion that the project's aim has indeed been met. It therefore appears that the problem was given sufficient consideration in the analysis and design phases to be able to be solved. Yet still it must be stressed that multiple improvements could still be made in the future, which would serve to enhance the user experience and increase the usability of the software, particularly by allowing remixes to be exported / shared.

\pagebreak
\subsection{Final Showcase / Evidence of Completion}
As alluded to in both the preface and testing sections, final evidence of the project's completion, for whom it may concern, can be viewed
\href{https://drive.google.com/file/d/1a68oN8o456MditAArxxx4ZTADfBsGJWR/view?usp=sharing}{here}\footnote{
	https://drive.google.com/file/d/1a68oN8o456MditAArxxx4ZTADfBsGJWR/view?usp=sharing
}, showcasing the fulfilment  of all the project's objectives. 